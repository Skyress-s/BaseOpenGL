\documentclass[14]{article}
\usepackage[utf8]{inputenc}

\usepackage{ulem}
\usepackage{amsmath}
\usepackage{amsfonts}
\usepackage{graphicx}


\usepackage[margin = 1.25in]{geometry}


\title{matteoblig 1}
\author{mathias mørk}
\date{january 2023}

\begin{document}
    \maketitle
    \begin{center}
    
    \section{introductionawd}

    \section{i like grass}
    
    
    test, here is some text {\large here is some large text}
    \large here is some more large text 
    \normalsize here we are back to normal text
    som more text
    \underline{underlined text}
    \uline{some better underlined text}
    
    this is a line \\ this is a new line \\[2\baselineskip] dobule break
    
    \section{displaymode math}
    awd
    \[f(x)=(x+2)^2 - 9\]    
    aw
    %* means not number
    \begin{align*} 
        & f(x)= bingus \\
        & hallo x +2^2 \\
         hee&eee
    \end{align*}
    
    \begin{align}
        & f(x) = x \\
        & some more text \\
        & aaaa wdad \nonumber \\ 
        & awawd \\
        & \left(  x*2 \right) \\
        \left(  x*2 \right) \\
        \mathbb{tau}
    \end{align}
        
    \section{inlinemode math}
    \(f(x) = x^2 + 4 \cdot x - 5 \)
    \(\frac{4}{3}\)
        
    
    \section{images}
    \graphicspath{{images/}}
    \includegraphics[width = 14 cm]{bingshilling}    
    \end{center}
\end{document}

