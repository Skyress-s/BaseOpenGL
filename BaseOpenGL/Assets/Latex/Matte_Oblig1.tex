\documentclass[14]{article}
\usepackage[utf8]{inputenc}

\usepackage{ulem}
\usepackage{amsmath}
\usepackage{amsfonts}
\usepackage{graphicx}


\usepackage[margin=1.25in]{geometry}


\title{Matte Oblig 1}
\author{Mathias Mohn Mørch}
\date{Januar 2023}

\begin{document}
    \maketitle
    \begin{flushleft}
        \graphicspath{{Images/}}

        \section{Innledning}
        I de følgende oppgavene skal vi bruke open GL sin grafikk funksjonalitet sammen med c++ for å løse og visualisere mattematiske grafen.
        Det skal også vises hvordan man gjør disse grapfene om til triangler som OpenGL kan tegne, samt åssen skrive og lese disse dataene fra fil.
        
        
        \section{Oppgave 1}
            \subsection*{1}
        Se vedlegg 1 for datafil.
        Valgte en \(h = \frac{1}{16}\) med en definisjons mendge \(\mathbb{D} = (0,0), (1,1)\)
        Brukte \(x^2\cdot y\) som funksjon\\
        kode snippet som lager grafen\\
        \includegraphics[width = 10cm]{ConstructTriangleSurface}
        
        Funksjoner som leser og skriver data
        
        \includegraphics[width = 12cm]{readWriteTriangle}
        \includegraphics[width=12cm]{readWriteTriangleDef}
        
        
            \subsection*{2}
        Lage partiellderiverte lambda funksjoner, og lage en til som regner ut normaler basert på disse
        
        \includegraphics[width=14cm]{functions}
        
        \includegraphics[width=14cm]{constructv2}
        
            \subsection*{3}
        Her brukte jeg en geometri shader steg for å tegne normal linjene basert på vertex dataen.
        Geometry Shader (Basert på learnopengl.com)
        
        \includegraphics[width=10cm]{geometryshader}
        
        Tegner først objectet med normal shader, derreter normal (som har geometry shader del) shader.
        
        \includegraphics[width=10cm]{doubledrawgeometrycode}
        
        Dette ga følgende resultat
        
        \includegraphics[width=8cm]{normalvectors}
        
        Oppengaven sa at man skulle modifisere draw funksjonen, så håper det er greit at jeg isteden bruker draw funksjonen to ganger.
        
        \section{Oppgave 2}
            \subsection*{1}
        Velger funksjonen \(f(x)=cos(4x) * 1(e^x)\), med definisjonsmengde \(\mathbb{D} = [0,5]\)
        
            \subsection*{2}
        Velger 20 intervaller, det gir en h på: 
        \begin{align*}
            & h = dx = \frac{b-a}{n}\\
            & h = \frac{5-0}{20} = 0.25
        \end{align*}
        
        \subsection*{3}
        Velger at fargen til vertexen skal være 
        \begin{align*}
            rgb = f(x)*0.5 + 0.5
        \end{align*}
        Dette er slik at alle punktene på kurven får en farge verdi mellom 0 og 1.
        
        \subsection*{4}
        Regnet ut funksjonsverdiene og pushet de til mVertices (arver fra Visual Object)   
        
        \includegraphics[width=14cm]{graph2dcode}
        
        Brukte derreter toFile funksjon (samme som i oppgave 1)
        
        \includegraphics[width=14cm]{graph2dtofilecode}
        
        \subsection*{5}
        
        \includegraphics[width=14cm]{lissajouseCurveCode}
        
        Valgte parameterene \(d = pi/2, a = 3, b = 4, A = 1, B = 1 og h = 0.1\) 
        Fikk følgende kurve
        
        \includegraphics[width=14cm]{lissajousecurve}
        
        
        \section{Oppgave 3}
        \subsection*{1}
        Funksjonene brukt
        
        \includegraphics[width=9cm]{numericintegral}
        
        Dette ga "total" lik \(0.16625 \text{. Fasit er } \frac{1}{6} = 0.16666...\) 
        
        \subsection*{2}
        
        Igjentok med halvert avstand \(h\), fikk resultatet \(0.166661)\)
        
        \section{Diskusjon}
        Generelt fant vi at datamaskinen, gitt liten nokk oppdelig \(h\), kan 
        med bra presisjon regne ut intergraler til grafer av to variable.
        I tilleg klarte vi å tegne flere grafer, både i 3d og 2d i et 3d rom med hjelp fra OpenGL
        
        Jeg har lært åssen man konstruerer en 3d flate med triagnler, hvor inndataen er en graf
        av to variable, samt mer åssen OpenGL generelt funker med sine VBO VBO osv.
        Itilleg har intergral oppgaven (Oppgave 3.1) hjulpet med å få en bedre intuitiv
        forståelse av åssen intergraler av funkksjoner av to variable fungerer.
        
        \section{Lenker}
        Github lenke : https://github.com/Skyress-s/BaseOpenGL/tree/Math3-Compulsory-1
    \end{flushleft}
\end{document}

