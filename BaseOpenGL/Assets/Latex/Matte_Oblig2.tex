\documentclass[14]{article}
\usepackage[utf8]{inputenc}

\usepackage{ulem}
\usepackage{amsmath}
\usepackage{amsfonts}
\usepackage{graphicx}

\usepackage[margin=1.25in]{geometry}
\usepackage{listings}

\title{Matte Oblig 1}
\author{Mathias Mohn Mørch}
\date{Februar 2023}

\begin{document}
    \maketitle
    \begin{flushleft}
        \graphicspath{{Images/}}

        \section{Innledning}
        I de følgende oppgave hvor vi skal nummerisk regne ut en polynom med minste kvadrats metode, 
        og regne ut et polynom men interpolasjon av 3 grad. \\
        Vi kommer også til å bruke openGL til å visualere polynomene vi får.
        

        \section{Oppgave 1}
        I denne oppgaven brukte jeg standardbasisen E.
        
        \subsection*{1}
        Bestemte 8 punkter for grafen.

        \begin{lstlisting}[language=C++, caption=C++ example]
    std::vector<glm::vec3> MathComp2Handler::GenerateRandomPoints() const {
        std::vector<glm::vec3> ps{};

        ps.push_back(glm::vec3(-4, 7, 0));
        ps.push_back(glm::vec3(-2.5, 4, 0));
        ps.push_back(glm::vec3(-2.3, 1, 0));
        ps.push_back(glm::vec3(-1.5, 0, 0));
        ps.push_back(glm::vec3(0, 2, 0));
        ps.push_back(glm::vec3(1, 3, 0));
        ps.push_back(glm::vec3(2, 5, 0));
        ps.push_back(glm::vec3(3, 7, 0));

        return ps;
    }
        \end{lstlisting}
        
        Skisee av punktene.

        \includegraphics[width=12cm]{task1_skisse}
        
        
        
        \subsection*{2}
        Brukte mattebibloteket Eigen til å skaffe of regne ut matrisene og løse ligningssystemet 
        \begin{align*}
            & A\cdot x = b \\
            & x = A^{-1} \cdot b
        \end{align*}
        
        
        
        \begin{lstlisting}[language=C++, caption=Funksjon som regner ut polynomet og lagrer numerisk grafdata]
void MathComp2Handler::HandleTask1() {
    points = GenerateRandomPoints();

    Eigen::MatrixXd A = Eigen::MatrixXd(points.size(), 3);
    A.setZero();
    for (int i = 0; i < points.size(); ++i) {
        A(i, 0) = pow(points[i].x, 2.f);
        A(i, 1) = points[i].x;
        A(i, 2) = 1.f;
    }
    std::cout << A << std::endl;
    Eigen::MatrixXd y = Eigen::MatrixXd(points.size(), 1);
    for (int i = 0; i < points.size(); ++i) {
        y(i, 0) = points[i].y;
    }
        
    std::cout << y << std::endl;
    
    Eigen::MatrixXd B = Eigen::Transpose<Eigen::MatrixXd>(A) * A;
    std::cout << B << std::endl;
    Eigen::MatrixXd c = Eigen::Transpose<Eigen::MatrixXd>(A) * y;

    Eigen::MatrixXd x = Eigen::Inverse<Eigen::MatrixXd>(B)*c;
    std::cout << "result" << std::endl << x << std::endl;
        
    // writing mVerices
    mVertices.clear();
    float min = -4.f;
    float max = 3.f;
    int divisions = 10;
    float delta = (max - min)/divisions;
        
    for (float i = min; i <= max+0.001f; i+=delta) {
        float y =
        x(0,0)*i*i +
        x(1,0)*i +
        x(2,0);
        Vertex v = Vertex(glm::vec3(i, y,0), glm::vec3(0,0,0));
        mVertices.push_back(v);    
    }
}
        \end{lstlisting}
        
        Dette ga oss følgene kolonne vektor
        \begin {center}
        $\begin{bmatrix}
            0.4975\\
            0.6513\\
            1.2513\\
        \end{bmatrix}$
        \end{center}
        
        Som gir oss følgende polynom
        
        \begin{align*}
            f(x) = 0.4975 \cdot  x^2 + 0.6513 \cdot x + 1.2513
        \end{align*}
        
        \subsection*{3}
        
        Skrev dette til fil, og leste inn i en Graph2D klasse

        (i main().)
        \begin{lstlisting}[language=C++, caption=main() del som lagrer klasse med oblig logic og 2d graf klasse som leser dataen]
    KT::MathComp2Handler* handler = new KT::MathComp2Handler(); 
    handler->HandleTask1();
    handler->ToFile("math2_2");
    mMap.insert(MapPair("math2", handler));

    KT::Graph2D* graph_2 = new KT::Graph2D(KT::FileHandler::VertexFromFile("math2_2"));
    mMap.insert(MapPair("graph_2", graph_2));

        \end{lstlisting}
        (ToFile og FromFile har blitt gjort i oblig1, velger derfor å ikke vise her.)
        
        Visualiseringen fra programmet
        
        \includegraphics[width=12cm]{opp1_visualisering}

        \section{Oppgave 2}
        I denne oppgaven brukte jeg standardbasisen E.
        \subsection*{1}
        
        Skrev inn punkter, fant \(A\) matrisen, lagde \(y\) matrise (y verdiene til punktene). 
        
        Regnet derreter ut løsningen av
        
        \begin{align*}
            & A\cdot x = b \\
            & x = A^{-1}\cdot b
        \end{align*}
        
        og skrev til fil.
        
        \begin{lstlisting}[language=C++, caption=Oppgave 2 numerisk utregning]
void MathComp2Handler::HandleTask2() {
    std::vector<glm::vec3> ps{};
    ps.push_back(glm::vec3(0,0,0));
    ps.push_back(glm::vec3(1,2,0));
    ps.push_back(glm::vec3(2,5,0));
    ps.push_back(glm::vec3(3,6,0));
    points = ps;

        
    Eigen::MatrixXd A = Eigen::MatrixXd(4, 4);
    for (int i = 0; i < ps.size(); ++i) {
        A(i,0) = ps[i].x * ps[i].x * ps[i].x;
        A(i,1) = ps[i].x * ps[i].x;
        A(i,2) = ps[i].x;
        A(i,3) = 1.0;
    }

    std::cout << A <<  std::endl;

    Eigen::Vector4d y{};
    for (int i = 0; i < ps.size(); ++i) {
        y(i) = ps[i].y;
    }
        
    std::cout << y << std::endl;
    Eigen::Vector4d x = A.inverse() * y;
    std::cout << x << std::endl;

    mVertices.clear();
    float min = 0.f;
    float max = 3.f;
    int divisions = 10;
    float delta = (max - min)/divisions;
    
    for (float i = 0; i <= max+0.001f; i+=delta) {
        float y =
            x(0)*i*i*i +
            x(1)*i*i +
            x(2)*i +
            x(3);
        Vertex v = Vertex(glm::vec3(i, y,0), glm::vec3(0,0,0));
        mVertices.push_back(v);    
    }
    
}
        \end{lstlisting}
        
        Kjørte programmet og dette ga følgende kolonne vektor
        
        \begin{align*}
            \begin{bmatrix}
                -0.5 \\ 2 \\ 0.5 \\ 0
            \end{bmatrix}
            \end{align*}
        
        Som ga oss polynomet 
        \begin{align*}
            & f(x) = -0.5 \cdot x^3 + 2 \cdot x^2 + 0.5 \cdot x + 0
        \end{align*}

        \subsection*{2}
        
        Brukte punktene regnet ut fra del 1 av oppgave 2 og leste fildata inn i en Graph2D klasse\\
        Gav følgene bilde i 3d program.
        
        \includegraphics[width=12cm]{task2_visualisering}
        

        \section{Diskusjon}
        Har lært mer om åssen man finner løsningen av matrise likningsystemer og hvordan dette kan gjøres nummerisk.
        
        
        Har også lært mer om komposisjon, dette er i hovedgrad grunnet at mange klasser trenger å skrive KT::Vertex
        data til fil. Da lærte jeg hvor nytting det kan være å lage en FileHandler klasse slik at ikke alle klasser trenger 
        å arve funksjonaliteten, og samt hvordan man bruker dette via komposisjon.
        
        Lærte også mer om hvordan interpolasjon fungerer. Grunnet at dette gir mer repitisjon av konseptene vi har lært i timen,
        samt at når man programmerer det må man ha "tunga rett i munnen".
       
        \section{Lenker}
        Github lenke : https://github.com/Skyress-s/BaseOpenGL/tree/Math3-Compulsory-2
    \end{flushleft}
\end{document}

